%% LyX 2.3.6 created this file.  For more info, see http://www.lyx.org/.
%% Do not edit unless you really know what you are doing.
\documentclass[10pt,spanish]{article}
\usepackage{mathpazo}
\usepackage{courier}
\usepackage[LGR,T1]{fontenc}
\usepackage[latin9]{inputenc}
\usepackage[a4paper]{geometry}
\geometry{verbose,tmargin=1.5cm,bmargin=1.5cm,lmargin=1.5cm,rmargin=1.5cm,headheight=0cm,headsep=0cm,footskip=0.7cm,columnsep=0.8cm}
\setlength{\parskip}{\medskipamount}
\setlength{\parindent}{0pt}
\usepackage{babel}
\addto\shorthandsspanish{\spanishdeactivate{~<>}}

\usepackage{amsbsy}
\usepackage{amssymb}
\usepackage{setspace}
\setstretch{1.2}
\usepackage[unicode=true,
 bookmarks=true,bookmarksnumbered=false,bookmarksopen=false,
 breaklinks=false,pdfborder={0 0 1},backref=false,colorlinks=false]
 {hyperref}

\makeatletter

%%%%%%%%%%%%%%%%%%%%%%%%%%%%%% LyX specific LaTeX commands.
\newcommand{\noun}[1]{\textsc{#1}}
%% Because html converters don't know tabularnewline
\providecommand{\tabularnewline}{\\}

\@ifundefined{date}{}{\date{}}
%%%%%%%%%%%%%%%%%%%%%%%%%%%%%% User specified LaTeX commands.
\usepackage{babel}







\renewcommand{\fnum@figure}{Figure~\thefigure}
\renewcommand{\fnum@table}{Table~\thetable}
\usepackage[margin=10pt,font=footnotesize,labelfont=bf,labelsep=period,justification=justified]{caption}
%justification=centering
\usepackage[skip=4pt]{caption}







\decimalpoint

\usepackage{titlesec}
%\titlespacing*{\section}{0pt}{3.5ex plus 1ex pt minus .2ex}{2.3ex plus .2ex}
%\titlespacing*{\subsection}{0pt}{3.25ex plus 1ex minus .2ex}{1.5ex plus .2ex}
%\titlespacing*{\subsection}{0pt}{3.25ex plus 1ex minus .2ex}{1.5ex plus .2ex}
\titlespacing*{\section}{0pt}{2.5ex plus .2ex minus .2ex}{0.5ex plus 0ex}
\titlespacing*{\subsection}{0pt}{2.5ex plus .2ex minus .2ex}{0.2ex plus 0ex}
\titlespacing*{\subsubsection}{0pt}{2.5ex plus .2ex minus .2ex}{0.1ex plus 0ex}

\setlength{\floatsep}{0.7\baselineskip plus 0pt minus 0pt}
\setlength{\textfloatsep}{0.7\baselineskip plus 0pt minus 0pt}
\setlength{\intextsep}{0.7\baselineskip plus 0pt minus 0pt}

\addtolength{\skip\footins}{5pt plus 0pt}

\let\@fnsymbol\@arabic

\@addtoreset{footnote}{page}
\renewcommand{\thefootnote}{\ifcase\value{footnote}\or*\or**\or***\or
\#\or\#\#\or\#\#\#\fi}

\usepackage{pdflscape}

\makeatother

\begin{document}
\begin{landscape}\par 
\begin{table}[h]
\caption{COMs and COM precursors transitions covered in our L1517B observations
and their derived line parameters.}

\begin{tabular}{ccccccccccc}
\hline 
 &  &  & \multicolumn{4}{c}{\textbf{Dust peak}} & \multicolumn{4}{c}{\textbf{Methanol peak}}\tabularnewline
\hline 
\textbf{Species} & \textbf{Line } & \textbf{Frequency } & \textbf{Area} $^{\mathrm{a}}$  & \textbf{Linewidth}  & \textbf{LSR velocity} $^{\mathrm{b}}$  & \textbf{S/N} $^{\mathrm{c}}$  & \textbf{Area} $^{\mathrm{a}}$  & \textbf{Linewidth}  & \textbf{LSR velocity} $^{\mathrm{b}}$  & \textbf{S/N} $^{\mathrm{c}}$\tabularnewline
 &  & (MHz)  & (mK km s$^{-1}$)  & (km s$^{-1}$)  & (km s$^{-1}$)  &  & (mK km s$^{-1}$)  & (km s$^{-1}$)  & (km s$^{-1}$)  & \tabularnewline
\hline 
--tablelines--  &  &  &  &  &  &  &  &  &  & \tabularnewline
\hline 
\end{tabular}

\textbf{\footnotesize{}Notes.}{\footnotesize{} Line profiles were
fitted using }\noun{\footnotesize{}Madcuba}{\footnotesize{}, except
for methanol ($\mathrm{CH_{3}OH}$), cyanoacetilene (HCCCN) and acetonitrile
($\mathrm{CH_{3}CN}$), where we used }\noun{\footnotesize{}Class}{\footnotesize{}
(see Section X for details). ($^{\mathrm{a}}$) Uncertainties in the
line area are calculated as $\Delta T(\Delta v\thinspace\delta v)^{1/2}$,
with $\Delta T$ the RMS noise level, $\Delta v$ the line width,
and $\delta v$ the velocity resolution of the spectrum. Similarly,
upper limits are calculated as 3 $\Delta T(\Delta v\thinspace\delta v)^{1/2}$.
($^{\mathrm{b}}$) LSR stands for }\emph{\footnotesize{}local sidereal
rest}{\footnotesize{}. ($^{\mathrm{c}}$) This refers to the signal
to noise ratio in integrated intensity area. If a certain value has
no uncertainty, this means that it had to be fixed so that }\noun{\footnotesize{}Madcuba}{\footnotesize{}
could fit the LTE model.}{\footnotesize\par}

 
\end{table}

\par \par \par \par \par 
\begin{table}[h]
\caption{Excitation/kinetic temperatures (\emph{$T$}), column densities (\emph{$N_{\mathrm{obs}}$}),
and abundances (\emph{$\chi_{\mathrm{obs}}\thinspace$}) of COMs and
COM precursors toward the dust and methanol peaks in L1517B.}

\begin{tabular}{ccccccc}
\hline 
 & \multicolumn{3}{c}{\textbf{Dust peak}} & \multicolumn{3}{c}{\textbf{Methanol peak}}\tabularnewline
\hline 
\textbf{Molecule} & \emph{$\boldsymbol{T}\thinspace(\mathrm{K})$}  & \emph{$\boldsymbol{N}_{\mathrm{\mathbf{obs}}}\thinspace(\mathrm{cm}{}^{-2})$}  & \emph{$\boldsymbol{\chi}_{\mathbf{\mathrm{\mathbf{obs}}}}\thinspace$}  & \emph{$\boldsymbol{T}\thinspace(\mathrm{K})$}  & \emph{$\boldsymbol{N}_{\mathrm{\mathbf{obs}}}\thinspace(\mathrm{cm}{}^{-2})$}  & \emph{$\boldsymbol{\chi}_{\mathrm{\mathbf{obs}}}\thinspace$}\tabularnewline
\hline 
--tableabunds--  &  &  &  &  &  & \tabularnewline
\hline 
\end{tabular}

\textbf{\footnotesize{}Notes.}{\footnotesize{} Temperatures ($T$)
refer to excitation temperatures ($T_{\mathrm{ex}}$ ) for all the
species except for methanol ($\mathrm{CH_{3}OH}$), cyanoacetilene
(HCCCN) and acetonitrile ($\mathrm{CH_{3}CN}$), where they refer
to kinetic temperatures ($T_{\mathrm{kin}}$ ). We used }\noun{\footnotesize{}Madcuba}{\footnotesize{}
to derive the molecular parameters from the observations except for
methanol, cyanoacetilene and acetonitrile, where we used }\noun{\footnotesize{}Radex}{\footnotesize{}.
Molecular abundances were calculated using an $\mathrm{H_{2}}$ column
density of $(3.5\pm0.5)\times10^{22}\thinspace\mathrm{cm}{}^{-2}$
for the dust continuum peak and of $(9.6\pm1.0)\times10^{21}\thinspace\mathrm{cm}{}^{-2}$
for the position of the methanol peak. For the non-detections and
also for some detections, we had to fix the excitation temperature
($T_{\mathrm{ex}}$ ) so that }\noun{\footnotesize{}Madcuba}{\footnotesize{}
could fit the column density ($N_{\mathrm{obs}}$ ). }{\footnotesize\par}

 
\end{table}

\par \par \par \par \par \par \end{landscape} 
\end{document}
